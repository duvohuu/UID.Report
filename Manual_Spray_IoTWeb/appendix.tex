\chapter*{PHỤ LỤC}
\addcontentsline{toc}{chapter}{PHỤ LỤC}
\fancyhead[R]{PHỤ LỤC}
\setcounter{figure}{0}
\setcounter{table}{0}
\thispagestyle{fancy}
\renewcommand{\thefigure}{A.\arabic{figure}}
\renewcommand{\thetable}{A.\arabic{table}}
    \section{Các linh kiện điện}
        \subsection[Cảm biến hồng ngoại TCRT5000]{Cảm biến hồng ngoại TCRT5000\footnote{Tài liệu tham khảo: \cite{vishay_tcrt5000_1}, trang 4}}
            \begin{figure}[H]
                \centering
                \includegraphics[width=0.88\textwidth]{pictures/appendix/app_p1_TCRT5000Dimensions1.png}
                \includegraphics[width=0.88\textwidth]{pictures/appendix/app_p2_TCRT5000Dimensions2.png}
                \caption{Cảm kích thước của cảm biến hồng ngoại TCRT5000}
                \label{fig:TCRT5000}
            \end{figure}
        \subsection[Tụ điện và cuộn cảm cho mạch hạ áp]{Tụ điện và cuộn cảm cho mạch hạ áp\footnote{Tài liệu tham khảo: \cite{lm2596}, trang 25}}
            \begin{figure}[H]
                \centering
                \includegraphics[width=1\textwidth]{pictures/appendix/app_p3_ChooseComponent.png}
                \caption{Cuộn cảm và tụ điện}
                \label{fig:InductorAndCapacitor}
            \end{figure}
        \subsection[Diode cho mạch hạ áp]{Diode cho mạch hạ áp\footnote{Tài liệu tham khảo: \cite{lm2596}, trang 26}}
            \begin{figure}[H]
                \centering
                \includegraphics[width=1\textwidth]{pictures/appendix/app_p4_DiodeChoose.png}
                \caption{Diode Schottky 1N5822}
                \label{fig:Diode}
            \end{figure}
    \newpage
    \section{Code}
    \begin{lstlisting}[caption={Đọc file nhị phân và plot sai số e2}, label={lst:e2_plot}]
        import numpy as np
        import matplotlib.pyplot as plt

        # Config
        file_path = "data_e2_blue_16_bits.bin" 
        dt = 0.05                     # Sample time
        data_type = np.float32        # data type in buffer

        # Read file data .bin
        with open(file_path, "rb") as f:
            raw = f.read()

        # Transform to numpy array
        values = np.frombuffer(raw, dtype=data_type)

        # Create time axis
        n = len(values)
        time = np.arange(0, n*dt, dt)

        # Total time
        t_max = 20
        max_samples = int(t_max / dt)  # Number of sample

        # Plot values
        values_plot = values[:max_samples]
        time_plot   = time[:max_samples]

        plt.figure(figsize=(10,5))
        plt.plot(time_plot, values_plot, marker='o', markersize=2, linestyle='-')
        plt.title("Line tracking error e2 measured over time")
        plt.xlabel("Time (s)")
        plt.ylabel("Error e2 (mm)")
        plt.grid(True)
        plt.tight_layout()
        plt.show()
    \end{lstlisting}
    \section{Biểu đồ Gantt}
        \subsection{Biểu đồ Gantt 1}
            \begin{figure}[H]
                \centering
                \includegraphics[width=0.8\textheight,angle=-90]{pictures/appendix/gantt1.png}
                \caption{Biểu đồ Gantt xây dựng từ đầu}
                \label{gantt1}
            \end{figure}
        \subsection{Biểu đồ Gantt 2}
            \begin{figure}[H]
                \centering
                \includegraphics[width=0.8\textheight,angle=-90]{pictures/appendix/gantt2.png}
                \caption{Biểu đồ Gantt thực tế triển khai}
                \label{gantt2}
            \end{figure}
        \subsection{Biểu đồ Gantt 3}
            \begin{figure}[H]
                \centering
                \includegraphics[width=0.8\textheight,angle=-90]{pictures/appendix/gantt3.png}
                \caption{Biểu đồ Gantt đề xuất}
                \label{gantt3}
            \end{figure}
        
