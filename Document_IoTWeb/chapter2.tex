\chapter{CẤU TRÚC HỆ THỐNG} 
    \section{Kiến trúc tổng thể}
        \begin{itemize}
            \item Front-end sử dụng React làm framework chính, Vite làm công cụ build và Material-UI cho UI components.
            \item Back-end sử dụng Node.js với Express để xây dựng RESTful API và MongoDB để lưu trữ dữ liệu.
            \item Giao tiếp IoT sử dụng giao thức MQTT để truyền dữ liệu thời gian thực.
        \end{itemize}
    \section{Kiến trúc triển khai (Deployment Architecture)}
        \hspace*{0.6cm}Sử dụng 3-Tier Architecture như sau:
        \begin{itemize}
            \item \textbf{Presentation Tier:} Giao diện người dùng (Front-end) được triển khai trên dịch vụ Vercel.
            \item \textbf{Logic Tier:} Máy chủ ứng dụng (Back-end) được triển khai trên Renbder, xử lý các yêu cầu từ giao diện người dùng và giao tiếp với cơ sở dữ liệu.
            \item \textbf{Data Tier:} Cơ sở dữ liệu MongoDB được lưu trữ trên dịch vụ MongoDB Atlas.
        \end{itemize}
        \begin{figure}[H]
            \centering
            \includegraphics[width=1\textwidth]{pictures/chapter2/c02_p01_DeploymentArchitecture.png}
            \caption{Kiến trúc triển khai hệ thống}
        \end{figure}
    \section{Kiến trúc phân lớp}
        \hspace*{0.6cm}Back-end sử dụng kiến trúc phân lớp (Layered Architecture) với 4 tầng chính:
        \begin{itemize}
            \item \textbf{Routing Layer:} Định nghĩa API endpoints và map HTTP methods (GET, POST, PUT, DELETE).
            \item \textbf{Controller Layer:} Xử lý HTTP request/response, validate input data và format JSON response.
            \item \textbf{Service Layer:} Business logic và data aggregation.
            \item \textbf{Model Layer:} Data access và database operations
        \end{itemize}
        \begin{figure}[H]
            \centering
            \begin{tikzpicture}[
                font=\small,
                node distance=1cm,
                box/.style={
                    draw,
                    rectangle,
                    rounded corners,
                    align=center,
                    inner sep=6pt,
                    text width=9.5cm
                },
                arrow/.style={->, thick}
            ]

            \node[box] (client) {
            \textbf{CLIENT REQUEST}\\
            \texttt{GET /api/machines}
            };

            \node[box, below=of client] (routing) {
            \textbf{LAYER 1: ROUTING}\\
            -- Định nghĩa endpoints\\
            -- Map HTTP methods
            };

            \node[box, below=of routing] (controller) {
            \textbf{LAYER 2: CONTROLLER}\\
            -- Xử lý request / response\\
            -- Validate input\\
            -- Gọi Service\\
            -- Format JSON response
            };

            \node[box, below=of controller] (service) {
            \textbf{LAYER 3: SERVICE}\\
            -- Business logic\\
            -- Role-based filtering\\
            -- Data transformation\\
            -- Gọi Model
            };

            \node[box, below=of service] (model) {
            \textbf{LAYER 4: MODEL}\\
            -- Mongoose Schema\\
            -- Database queries\\
            };

            \node[box, below=of model] (db) {
            \textbf{DATABASE}\\
            MongoDB Atlas
            };

            \node[box, below=of db] (response) {
            \textbf{RESPONSE TO CLIENT}\\
            \texttt{\{ success, count, machines \}}
            };

            \draw[arrow] (client) -- (routing);
            \draw[arrow] (routing) -- (controller);
            \draw[arrow] (controller) -- (service);
            \draw[arrow] (service) -- (model);
            \draw[arrow] (model) -- (db);
            \draw[arrow] (db) -- (response);

            \end{tikzpicture}
            \caption{Luồng xử lý request theo kiến trúc nhiều lớp của Backend}
        \end{figure}
    \section{Mô hình MVC}
        \hspace*{0.6cm}Mô hình MVC (Model-View-Controller) là một mô hình thiết kế phần mềm được sử dụng phổ biến trong phát triển ứng dụng web. Mô hình này chia ứng dụng thành ba phần: Model (mô hình), View (giao diện) và Controller (bộ điều khiển). Mỗi phần có nhiệm vụ riêng và tương tác với nhau để tạo ra ứng dụng hoàn chỉnh.
            \begin{figure}[H]
                \centering
                \includegraphics[width=0.8\textwidth]{pictures/chapter2/c02_p02_MVC.png}
                \caption{Mô hình MVC}
                \label{fig:mvc}
            \end{figure}
        Front-end sử dụng mô hình MVC kết hợp Custom Hooks Pattern như sau:
        \begin{figure}[H]
            \centering
            \begin{tikzpicture}[
                font=\small,
                node distance=1cm,
                box/.style={
                    draw,
                    rectangle,
                    rounded corners,
                    align=center,
                    inner sep=6pt,
                    text width=8cm
                },
                arrow/.style={->, thick}
            ]

            % ================= VIEW LAYER =================
            \node[box] (view) {
            \textbf{VIEW LAYER (UI)}\\
            Pages \& Components (React JSX)\\[6pt]
            };

            % ================= CONTROLLER LAYER =================
            \node[box, below=of view] (controller) {
            \textbf{CONTROLLER LAYER (Logic)}\\
            Custom Hooks (Business Logic)\\[6pt]
            };

            % ================= MODEL LAYER =================
            \node[box, below=of controller] (model) {
            \textbf{MODEL LAYER (Data)}\\
            API Services \& Global State\\[6pt]
            };

            % ================= BACKEND =================
            \node[box, below=of model] (backend) {
            \textbf{BACKEND API}\\
            RESTful API + Socket.IO
            };

            % ================= ARROWS =================
            \draw[arrow] (view) -- node[right]{Props \& Events} (controller);
            \draw[arrow] (controller) -- node[right]{API Calls} (model);
            \draw[arrow] (model) -- node[right]{HTTP / WebSocket} (backend);

            \end{tikzpicture}
            \caption{Kiến trúc Frontend theo mô hình View--Controller--Model và kết nối Backend}
        \end{figure}
    \section{Sơ đồ luồng dữ liệu (Data Flow Diagram)}
        \begin{figure}[H]
            \centering
            \begin{tikzpicture}[
                font=\small,
                node distance=1.2cm,
                box/.style={
                    draw,
                    rectangle,
                    rounded corners,
                    align=center,
                    inner sep=6pt,
                    text width=9.5cm
                },
                arrow/.style={->, thick}
            ]

            % ========== IoT Device ==========
            \node[box] (iot) {
            \textbf{IoT DEVICE}\\
            Spray Machine Physical Hardware\\[4pt]
            \textbf{Sensors}\\
            Status (0/1), Power Consumption (kWh)
            };

            % ========== MQTT Broker ==========
            \node[box, below=of iot] (broker) {
            \textbf{MQTT BROKER (HiveMQ)}\\
            \texttt{broker.hivemq.com:1883}\\[4pt]
            Message Queue \& Distribution Service
            };

            % ========== Backend ==========
            \node[box, below=of broker] (backend) {
            \textbf{BACKEND SERVER (Node.js + Express)}\\[4pt]
            MQTT Client $\rightarrow$ Machine Service $\rightarrow$ Spray Machine Service\\
            $\rightarrow$ Data Processing $\rightarrow$ Database (MongoDB)
            };

            % ========== Frontend ==========
            \node[box, below=of backend] (frontend) {
            \textbf{FRONTEND CLIENT (React + Vite)}\\[4pt]
            Socket Context $\rightarrow$ Hooks $\rightarrow$ State $\rightarrow$ Charts
            };

            % ===== Arrows =====
            \draw[arrow] (iot) -- node[right]{MQTT Publish} (broker);
            \draw[arrow] (broker) -- node[right]{MQTT Subscribe} (backend);
            \draw[arrow] (backend) -- node[right, align=center]{WebSocket\\Real-time} (frontend);

            \end{tikzpicture}
            \caption{Luồng giao tiếp dữ liệu IoT -- Backend -- Frontend của hệ thống}
        \end{figure}